\documentclass{article}
\usepackage[utf8]{inputenc}
\usepackage[brazil]{babel}
\usepackage[a4paper, left=20mm, right=20mm, top=20mm, bottom=20mm]{geometry}
\usepackage[colorlinks, urlcolor=blue, citecolor=red]{hyperref}


\title{\textbf{Atividade T2 - INE5430 - Inteligência Artificial}}
\author{
    Caique Rodrigues Marques \\
    {\texttt{c.r.marques@grad.ufsc.br}}
    \and
    Fernando Jorge Mota \\
    {\texttt{contato@fjorgemota.com}}
    \vspace{-5mm}
}
\date{09 de setembro de 2016}

\begin{document}

\maketitle

\section*{Questão 1}
    Desde o início da ciência e da filosofia, houve a dúvida de como o
    conhecimento é e poderia ser moldado. Em 1957 surge o conceito de redes
    semânticas, por Richard Richens, um ramo da IA forte - uma das bases da
    representação de orientação a objetos. Na década de 1960, a linguagem de
    programação Lisp era bastante usada no ramo da inteligência artificial, um
    dos recursos que a linguagem oferecia era o uso de itens (ou átomos), que
    são estruturas que continham atributos e eram imutáveis, podendo associar
    entre si. Nesta mesma época surgia a linguagem de programação SIMULA 67,
    criada por Ole-Johan Dahl e Kristen Nygaard, que foi a primeira linguagem
    de programação orientada a objetos e base para o popular Smalltalk e as
    linguagens de programação modernas orientadas a objetos. Em 1975, Marvin
    Minsky mencionou que a principal base para o conceito de Frames, que ele
    mesmo propôs, foi a orientação a objetos.

\section*{Questão 2A}
    O tipo de conhecimento representado com os sistemas de produção e sistemas
    especialistas é um tipo de conhecimento bem especifico a respeito de uma
    determinada área qualquer especificada geralmente (mas não necessariamente,
    no caso dos sistemas especialistas) pelas regras - chamadas normalmente de
    produções - que o sistema possui para realizar seu processamento. 
    
    Tais regras são a principio informadas ao sistema por especialistas de
    domínio em ambos os casos, e constituem-se de condições "\texttt{SE ...
    ENTÃO ...}" que permitem à máquina estabelecer relações claras entre as
    diferentes regras para formar novos conhecimentos. Nos casos em que ambas
    os sistemas são baseados em regras, há diferenças nos métodos de inferência
    usados para formar novos fatos, pois no caso dos sistemas de produção é
    usado "encadeamento para frente" e nos sistemas especialistas podem ser
    usados tanto "encadeamento para frente" quanto "encadeamento para trás".
    
    O que fica bem estabelecido, mesmo, a respeito de ambos os sistemas, é que
    em ambos o conhecimento inicial é diretamente inserido por especialistas -
    numa fase normalmente chamada de aquisição de conhecimentos - e que o
    sistema é capaz de trabalhar, dado esse conhecimento inicial, apenas em
    cima dele, sem se estender para outras áreas ou conhecimentos de propósito
    mais geral, por exemplo.

\section*{Questão 2B}
    Sistemas Especialistas e Sistemas de Produção não são, segundo nossas
    pesquisas, mais tão usados quanto eram na década de 90. Ainda assim, esses
    sistemas ainda são mencionados periodicamente em livros e conferências na
    área de inteligência artificial, sem citar aplicações ou implementações
    significativas relacionadas a esses sistemas.
    
    Das aplicações que encontramos que mencionam esses sistemas, muitas afirmam
    não usá-los por se tratar de problemas no qual sistemas especialistas e
    sistemas de produção necessitariam de uma grande quantidade de regras para
    funcionar da forma esperada. 
    
    O único documento que encontramos que falam de um uso recente de sistemas
    especialistas trata da implementação de \textit{human mental workload} (ou
    carga de trabalho mental humana, em tradução livre) usando sistemas
    especialistas como uma forma mais estruturada de implementar tal carga de
    trabalho. Já na maioria dos casos, como no caso de predição de
    \textit{phishing} usando inteligência artificial, foi observado que não foi
    usado sistemas especialistas devido ao fato de que tais sistemas
    necessitariam de grande quantidade de regras para realizar o trabalho, daí
    justificando o uso de outros mecanismos de representação de reconhecimento
    para realizar a tarefa.
    
    A respeito de sistemas de produção, encontrou-se grande dificuldade em
    encontrar documentos relacionados a área devido a confusão gerada pelo nome
    do sistema, que em muitos documentos aparece sob o contexto de "colocar um
    sistema em produção" - mesmo quando buscando por tópicos relacionados a
    inteligência artificial - o que claramente não corresponde ao que queríamos
    encontrar durante a pesquisa.

\section*{Questão 3}
    No ramo de inteligência artificial fraca, ramo mais voltado a como a
    máquina pode simular comportamentos humanos, há ferramentas que ligam a
    representação do conhecimento à IA fraca. Por exemplo, redes semânticas
    surgiram como uma forma da máquina reconhecer a linguagem natural humana,
    que é uma aplicação específica, através de modelo de grafos e hoje é usada,
    dentre as várias aplicações, para reconhecimento de plágio. Neste caso o
    conhecimento é moldado numa rede de elementos-chaves que são
    interconectados através de referências entre cada elemento-chave - isto
    também é uma forma de reconhecimento de ideias, visto que normalmente
    associamos os assuntos que estamos aprendendo a algo familiar que já temos
    conhecimento. Outra ferramenta específica que moldou a representação de
    conhecimento foi o Logic Theorist, em 1955, conseguiu provar diversos
    teoremas matemáticos (já provados) apenas usando a lógica, conseguindo até
    achar provas alternativas aos já conhecidos; lógica é um conceito que
    usamos para para embasamento de ideias.

\section*{Referências}
    \subsection*{Questão 1}
    \begin{itemize}
        %% \href{[url]}{[texto]}
        \item
            \href{https://books.google.com.br/books?id=OKn2v__x-OwC&lpg=PP1&dq=Operations%20Research%20and%20Artificial%20Intelligence&hl=pt-BR&pg=PP1#v=onepage&q=Operations%20Research%20and%20Artificial%20Intelligence&f=false}{C.
            W. Holsapple, Varghese S. Jacob, Andrew B. Whinston;
            \textit{Operations Research and Artificial Intelligence}, Ablex
            Publising Corporation, 1994 (p. 59-60)}
        \item \href{https://en.wikipedia.org/wiki/Object-oriented_programming}{Object Oriented Programming - Wikipedia, the free encyclopedia}
        \item \href{https://en.wikipedia.org/wiki/Simula}{Simula - Wikipedia, the free encyclopedia}
        \item \href{https://en.wikipedia.org/wiki/Sketchpad}{Sketchpad - Wikipedia, the free encyclopedia}
        \item \href{http://web.stanford.edu/class/cs227/Lectures/lec02.pdf}{2. Object Oriented Representaion - CS227, Spring 2011 - Stanford University (slide 7)}
    \end{itemize}
    
    \subsection*{Questão 2}
    \subsubsection*{Questão A}
    \begin{itemize}
        \item \href{http://www.j-paine.org/students/tutorials/ps/node1.html}{Introduction --- the birth of expert systems and production systems}
        \item \href{https://en.wikipedia.org/wiki/Production_system_(computer_science)}{Production system (computer science) - Wikipedia, the free encyclopedia}
        \item \href{https://en.wikipedia.org/wiki/Expert_system}{Expert system - Wikipedia, the free encyclopedia}
    \end{itemize}
    
    \subsubsection*{Questão B}
    \begin{itemize}
        \item \href{https://books.google.com.br/books?id=GS7RDAAAQBAJ&lpg=PA80&dq=artificial%20intelligence%20expert%20system%202016&hl=pt-BR&pg=PA80#v=onepage&q=artificial%20intelligence%20expert%20system%202016&f=false}{Combining RDR-Based Machine Learning Approach and Human Expert Knowledge for Phishing Detection}
        \item \href{https://books.google.com.br/books?id=jYb2DAAAQBAJ&lpg=PA215&dq=artificial%20intelligence%20expert%20system%202016&hl=pt-BR&pg=PA215#v=onepage&q=artificial%20intelligence%20expert%20system%202016&f=false}{Modeling Mental Worload via Rule-Based Expert System: A Comparison with NASA-TLX and Workload Profile}
        
    \end{itemize}
\end{document}
