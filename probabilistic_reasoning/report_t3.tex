\documentclass{article}
\usepackage[utf8]{inputenc}
\usepackage[brazil]{babel}
\usepackage[a4paper, left=20mm, right=20mm, top=20mm, bottom=20mm]{geometry}
\usepackage[fleqn]{amsmath}

\title{\textbf{INE5430 - Inteligência Artificial \\
        \large Trabalho T3 - Raciocínio Probabilístico}}
\author{
    Caique Rodrigues Marques \\
    {\texttt{c.r.marques@grad.ufsc.br}}
    \and
    Fernando Jorge Mota \\
    {\texttt{contato@fjorgemota.com}}
    \vspace{-5mm}
}
\date{}
\begin{document}
    \maketitle
    \section*{Questão 1}
        Qual a probabilidade de não ver um zumbi dado que um feitiço
        mágico não foi lançado e não há um surto viral? Pelas informações
        fornecidas, é possível inferir:
        \begin{gather*}
            P(Zb=N|Vr=N \land Mg=N) = 0.99 = 99\%
        \end{gather*}

    \section*{Questão 2}
        Qual a probabilidade do mundo estar no seguinte estado: 
        $P(Mg=S, Hp=S, Zb=S, Vr=S, Tp=S)$? \\
        Tal estado significa todas as possíveis aparições dos seres e dos
        eventos ao mesmo tempo, portanto:
        \begin{gather*}
            P(Mg=S, Hp=S, Zb=S, Vr=S, Tp=S) \\
            \,= P(Mg=S) \times P(Vr=S) \times P(Hp=S|Mg=S) \\
            \qquad\times P(Zb=S|Mg=S \land Vr=S) \times P(Tp=S|Vr=S) \\
            \,= 0.1 \times 0.2 \times 0.8 \times 0.6 \times 0.3
            = 0.00288 \approx 0.288\%
        \end{gather*}

    \section*{Questão 3}
        Qual a probabilidade de se ver um zumbi? \\
        Como a aparição de um zumbi está relacionada com as aparições de um
        feitiço mágico e de um surto viral, estas variáveis \textit{a priori}
        também devem ser consideradas:
        \begin{gather*}
            P(Zb=S) \\
            \,= P(Zb=S|Mg=S \land Vr=S) \times P(Mg=S) \times P(Vr=S) \\
            \qquad+ P(Zb=S|Mg=S \land Vr=N) \times P(Mg=S) \times P(Vr=S) \\
            \qquad+ P(Zb=S|Mg=N \land Vr=S) \times P(Mg=N) \times P(Vr=S) \\
            \qquad+ P(Zb=S|Mg=N \land Vr=N) \times P(Mg=N) \times P(Vr=N) \\
            \,= (0.6 \times 0.1 \times 0.2) + (0.5 \times 0.1 \times 0.8)
            + (0.4 \times 0.9 \times 0.2) + (0.01 \times 0.9 \times 0.8) \\
            \,= 0.012 + 0.04 + 0.072 + 0.0072
            = 0.1312 = 13.12\%
        \end{gather*}

    \section*{Questão 4}
        Qual a probabilidade de se ver um zumbi dado que está havendo
        um surto viral? \\
        A aparição de um zumbi está relacionada com a aparição de um surto viral,
        portanto, há de se considerar a probabilidade de ambas acontecerem ao mesmo
        tempo.
        \begin{gather*}
            P(Zb=S \land Vr=S) \\
            \,= P(Zb=S|Mg=S \land Vr=S) \times P(Mg=S) \times P(Vr=S) \\
            \qquad+ P(Zb=S|Mg=N \land Vr=S) \times P(Mg=N) \times P(Vr=S) \\
            \,= (0.6 \times 0.1 \times 0.2) + (0.4 \times 0.9 \times 0.2)
            = 0.012 + 0.072 = 0.084
        \end{gather*}
        Portanto, pelo teorema de Bayes:
        \begin{gather*}
            P(Zb=S|Vr=S) = \frac{P(Zb=S \land Vr=S)}{P(Vr=S)} = \frac{0.084}{0.2}
            = 0.42 = 42\%
        \end{gather*}
    \section*{Questão 5}
        Qual a probabilidade de se ver um hipogrifo dado que você
        conseguiu ver um zumbi? \\
        Tanto a aparição de zumbi quanto a aparição de hipogrifo estão relacionados
        com o uso de feitiço mágico, assim, o caso em que zumbi e hipogrifo aparecem
        quando há o uso de feitiço mágico deve ser calculado:
        \begin{gather*}
            P(Hp=S \land Zb=S \land Mg=S) \\
            \,= P(Mg=S) \times P(Zb=S|Mg=S) \times P(Hp=S|Mg=S) \\
            \,= 0.1 \times 0.52 \times 0.8 = 0.0416
        \end{gather*}
        \begin{gather*}
            P(Zb=S|Mg=S) \\
            \,= P(Zb=S|Mg=S \land Vr=S) \times P(Vr=S)
            + P(Zb=S|Mg=S \land Vr=N) \times P(Vr=N) \\
            \,= (0.6 \times 0.2) + (0.5 \times 0.8)
            = (0.12) + (0.4) = 0.52
        \end{gather*}
        Inclusive, o caso dos avistamentos de zumbi e hipogrifo e o não uso de
        feitiço mágico deve ser considerado:
        \begin{gather*}
            P(Hp=S \land Zb=S \land Mg=N) \\
            \,= P(Mg=N) \times P(Zb=S|Mg=N) \times P(Hp=S|Mg=N) \\
            \,= 0.9 \times 0.088 \times 0.7 = 0.05544
        \end{gather*}
        \begin{gather*}
            P(Zb=S|Mg=N) \\
            \,= P(Zb=S|Mg=N \land Vr=S) \times P(Vr=S)
            + P(Zb=S|Mg=N \land Vr=N) \times P(Vr=N) \\
            \,= (0.4 \times 0.2) + (0.01 \times 0.8)
            = (0.08) + (0.008) = 0.088
        \end{gather*}
        Por fim, pelo teorema de Bayes:
        \begin{gather*}
            P(Hp=S|Zb=S) \\
            \,= \frac{P(Hp=S \land Zb=S \land Mg=S)
                + P(Hp=S \land Zb=S \land Mg=N)}{P(Zb=S)} \\
            \,= \frac{0.0416 + 0.05544}{0.1312}
            = \frac{0.09704}{0.1312}
            = 0.73963414 \approx  73.97\%
        \end{gather*}

    \section*{Questão 6}
        Qual a probabilidade de se ver um zumbi dado que você conseguiu
        ver um hipogrifo? \\
        Primeiramente, é necessário verificar a probabilidade de se ver
        um hipogrifo:
        \begin{gather*}
            P(Hp=S) \\
            \,= P(Hp=S|Mg=S) \times P(Mg=S) + P(Hp=S|Mg=N) \times P(Mg=N) \\
            \,= (0.8 \times 0.1) + (0.7 \times 0.9)
            \,= 0.08 + 0.63 = 0.71
        \end{gather*}
        Assim:
        \begin{gather*}
            P(Zb=S|Hp=S) \\
            \,= \frac{P(Hp=S \land Zb=S \land Mg=S \land Vr=S)
            + P(Hp=S \land Zb=S \land Mg=S \land Vr=N)}{P(Hp=S)} \\
            \qquad+ \frac{P(Hp=S \land Zb=S \land Mg=N \land Vr=S)
            + P(Hp=S \land Zb=S \land Mg=N \land Vr=N)}{P(Hp=S)} \\
            \,= \frac{P(Zb=S|Vr=S \land Mg=S) \times P(Mg=S) \times P(Vr=S)
                \times P(Hp=S|Mg=S)}{P(Hp=S)} \\
            \qquad+ \frac{P(Zb=S|Vr=N \land Mg=S) \times P(Mg=S) \times P(Vr=N)
                \times P(Hp=S|Mg=S)}{P(Hp=S)} \\
            \qquad+ \frac{P(Zb=S|Vr=S \land Mg=N) \times P(Mg=N) \times P(Vr=S)
                \times P(Hp=S|Mg=N)}{P(Hp=S)} \\
            \qquad+ \frac{P(Zb=S|Vr=N \land Mg=N) \times P(Mg=N) \times P(Vr=N)
                \times P(Hp=S|Mg=N)}{P(Hp=S)} \\
            \,= \frac{0.6 \times 0.1 \times 0.2 \times 0.8}{P(Hp=S)}
            + \frac{0.5 \times 0.1 \times 0.8 \times 0.8}{P(Hp=S)}
            + \frac{0.4 \times 0.9 \times 0.2 \times 0.7}{P(Hp=S)}
            + \frac{0.01 \times 0.9 \times 0.8 \times 0.7}{P(Hp=S)} \\
            \,= \frac{0.0096 + 0.032 + 0.0504 + 0.00504}{0.71}
            \,= \frac{0.09704}{0.71} = 0.136676056 \approx 13.67\%
        \end{gather*}

    \section*{Questão 7}
        Qual a probabilidade de se ver um zumbi dado que você
        conseguiu ver um hipogrifo e um viajante do tempo?
        Dado que as aparições de hipogrifo e viajante do tempo
        são eventos independentes (lembrando que já sabemos a
        probabilidade \textit{a priori} de aparição de hipogrifos,
        vide questão 6), portanto:
        \begin{gather*}
            P(Hp=S \land Tp=S) \\
            \,= P(Hp=S) \times P(Tp=S) \\
            \,= (0.71) \times 
                (P(Tp=S|Vr=S) \times P(Vr=S) + P(Tp=S|Vr=N) \times P(Vr=N)) \\
            \,= (0.71) \times (0.3 \times 0.2 + 0.1 \times 0.8)
                = 0.71 \times 0.14 = 0.0994
        \end{gather*}
        Assim:
        \begin{gather*}
            P(Zb=S|Hp=S \land Tp=S) \\
            \,= \frac{P(Zb=S \land Hp=S \land Tp=S \land Mg=S \land Vr=S)}
                {P(Hp=S \land Tp=S)} \\
            \qquad+ \frac{P(Zb=S \land Hp=S \land Tp=S \land Mg=S \land Vr=N)}
                {P(Hp=S \land Tp=S)} \\
            \qquad+ \frac{P(Zb=S \land Hp=S \land Tp=S \land Mg=N \land Vr=S)}
                {P(Hp=S \land Tp=S)} \\
            \qquad+ \frac{P(Zb=S \land Hp=S \land Tp=S \land Mg=N \land Vr=N)}
                {P(Hp=S \land Tp=S)} \\
            \,= \frac{P(Zb=S|Mg=S \land Vr=S) \times P(Mg=S) \times P(Vr=S)
                \times P(Hp=S|Mg=S) \times P(Tp=S|Vr=S)}{P(Hp=S \land Tp=S)} \\
            + \frac{P(Zb=S|Mg=S \land Vr=N) \times P(Mg=S) \times P(Vr=N)
                \times P(Hp=S|Mg=S) \times P(Tp=S|Vr=N)}{P(Hp=S \land Tp=S)} \\
            + \frac{P(Zb=S|Mg=N \land Vr=S) \times P(Mg=N) \times P(Vr=S)
                \times P(Hp=S|Mg=N) \times P(Tp=S|Vr=S)}{P(Hp=S \land Tp=S)} \\
            + \frac{P(Zb=S|Mg=N \land Vr=N) \times P(Mg=N) \times P(Vr=N)
                \times P(Hp=S|Mg=N) \times P(Tp=S|Vr=N)}{P(Hp=S \land Tp=S)} \\
            \,= \frac{0.6 \times 0.1 \times 0.2 \times 0.8 \times 0.3}
                {P(Hp=S \land Tp=S)}
            + \frac{0.5 \times 0.1 \times 0.8 \times 0.8 \times 0.1}
                {P(Hp=S \land Tp=S)} \\
            \qquad+ \frac{0.4 \times 0.9 \times 0.2 \times 0.7 \times 0.3}
                {P(Hp=S \land Tp=S)}
            + \frac{0.01 \times 0.9 \times 0.8 \times 0.7 \times 0.1}
                {P(Hp=S \land Tp=S)} \\
            \,= \frac{0.00288 + 0.0032 + 0.01512 + 0.000504}{0.0994}
            = \frac{0.021704}{0.0994} = 0.21835010060 \approx 21.84\%
        \end{gather*}
\end{document}