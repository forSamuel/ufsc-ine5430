\documentclass{article}
\usepackage[utf8]{inputenc}
\usepackage[brazil]{babel}
\usepackage[a4paper, left=20mm, right=20mm, top=20mm, bottom=20mm]{geometry}
\usepackage[colorlinks, urlcolor=blue, citecolor=red]{hyperref}
\usepackage{forest}
\usepackage{minted}


\title{\textbf{Atividade T1 - INE5430 - Inteligência Artificial}}
\author{
    Caique Rodrigues Marques \\
    {\texttt{c.r.marques@grad.ufsc.br}}
    \and
    Fernando Jorge Mota \\
    {\texttt{contato@fjorgemota.com}}
    \vspace{-5mm}
}
\date{02 de setembro de 2016}

\begin{document}

\maketitle

\section*{Questão 1}
    Como é de se esperar para o algoritmo do minimax com podas alfa e beta,
    temos que, inicialmente, vai ser associado, primeiramente, a todos os nodos
    os valores $\alpha=-\infty$ e $\beta=\infty$. Depois disso, é considerado o
    algoritmo do minimax e configurado os valores $\alpha$ e $\beta$ de forma a
    permitir descobrir que nodos podem ser cortados ou não. Para facilitar,
    resolvemos separar em duas subseções os valores obtidos após aplicar o
    algoritmo (no caso, os valores finais dos nodos e os valores alfa e beta
    associados a cada nodo). Boa leitura!

    \subsection*{Valores finais dos nodos}
        A árvore abaixo representa os valores finais dos nodos. Os nodos
        marcados com um "X" representa que houve uma poda na aresta
        correspondente. \\
        \begin{forest}
            for tree={
                align=center,
                l sep=0.5cm,
                s sep=0.1cm,
                minimum height=0.8cm,
                minimum width=1cm,
                draw
                }
              [
                5
                [ 
                  5
                  [
                    3
                    [ 
                      5
                      [
                        5
                       ]
                      [
                        2
                       ]
                     ]
                    [ 
                      2
                      [
                        9
                      ]
                      [
                        2
                      ]
                      [
                        X
                      ]
                    ]
                 ]
                 [
                   7
                   [
                     7
                     [
                       7
                     ]
                   ]
                 ]
               ]
               [
                 3
                 [
                   9
                   [
                     7
                     [
                       7
                     ]
                   ]
                   [
                     9
                     [
                       9
                     ]
                     [
                       10
                     ]
                   ]
                 ]
                 [
                   3
                   [
                     3
                     [
                       3
                     ]
                   ]
                 ]
               ]
               [
                 2
                 [
                   2
                   [
                     2
                     [
                       2
                     ]
                   ]
                 ]
                 [
                   X
                   [
                     X
                     [ 
                       X
                     ]
                     [
                       X
                     ]
                   ]
                   [
                     X
                     [
                       X
                     ]
                   ]
                 ]
               ]
             ]
        \end{forest}
    \subsection*{Valores Alfa e Beta}
        A árvore abaixo representa os valores alfa e beta na árvore de acordo
        com o tempo. Os valores mais à direita representam o valor final. Os
        nós com o valor "X" representam nós que foram podados.
              
        \hspace{-25mm}
        \begin{forest}
        for tree={
            align=center,
            l sep=0.4cm,
            s sep=0.1cm,
            font=\scriptsize,
            minimum height=0.6cm,
            minimum width=0.8cm,
            draw
            }
         [
           {$\alpha=\{-\infty, 5\}$ \\ $\beta=\{\infty\}$}
           [ 
             {$\alpha=\{-\infty\}$ \\ $\beta=\{\infty, 5\}$}
             [
               {$\alpha=\{-\infty, 5\}$ \\ $\beta=\{\infty\}$}
               [ 
                 {$\alpha=\{-\infty\}$ \\ $\beta=\{\infty, 7, 5\}$}
                 [
                   {$\alpha=\{-\infty\}$ \\ $\beta=\{\infty\}$}
                 ]
                 [
                   {$\alpha=\{-\infty\}$ \\ $\beta=\{\infty\}$}
                 ]
               ]
               [ 
                 {$\alpha=\{5\}$ \\ $\beta=\{\infty, 9, 2\}$}
                 [
                   {$\alpha=\{-\infty\}$ \\ $\beta=\{\infty\}$}
                 ]
                 [
                   {$\alpha=\{-\infty\}$ \\ $\beta=\{\infty\}$}
                 ]
                 [
                   X
                 ]
               ]
            ]
            [
              {$\alpha=\{-\infty, 7\}$ \\ $\beta=\{5\}$}
              [
                {$\alpha=\{-\infty\}$ \\ $\beta=\{5\}$}
                [
                  {$\alpha=\{-\infty\}$ \\ $\beta=\{\infty\}$}
                ]
              ]
            ]
          ]
          [
            {$\alpha=\{5\}$ \\ $\beta=\{\infty, 9\}$}
            [
              {$\alpha=\{5, 7, 9\}$ \\ $\beta=\{\infty\}$}
              [
                {$\alpha=\{5\}$ \\ $\beta=\{\infty, 7\}$}
                [
                  {$\alpha=\{-\infty\}$ \\ $\beta=\{\infty\}$}
                ]
              ]
              [
                {$\alpha=\{7\}$ \\ $\beta=\{\infty, 9\}$}
                [
                  {$\alpha=\{-\infty\}$ \\ $\beta=\{\infty\}$}
                ]
                [
                  {$\alpha=\{-\infty\}$ \\ $\beta=\{\infty\}$}
                ]
              ]
            ]
            [
              {$\alpha=\{5\}$ \\ $\beta=\{9\}$}
              [
                {$\alpha=\{5\}$ \\ $\beta=\{9, 3\}$}
                [
                  {$\alpha=\{-\infty\}$ \\ $\beta=\{\infty\}$}
                ]
              ]
            ]
          ]
          [
            {$\alpha=\{5\}$ \\ $\beta=\{\infty, 2\}$}
            [
              {$\alpha=\{5\}$ \\ $\beta=\{\infty\}$}
              [
                {$\alpha=\{5\}$ \\ $\beta=\{\infty, 2\}$}
                [
                  {$\alpha=\{-\infty\}$ \\ $\beta=\{\infty\}$}
                ]
              ]
            ]
            [
              X
              [
                X
                [ 
                  X
                ]
                [
                  X
                ]
              ]
              [
                X
                [
                  X
                ]
              ]
            ]
          ]
        ]
        \end{forest}

\section*{Questão 2}
    Para definir uma boa função heuristica e uma boa função utilidade para o
    jogo de damas, precisamos definir as condições que cada função deve
    considerar e isso depende do objetivo de cada função.

    Sabemos que, para o caso da função heurística, temos como meta determinar
    \textbf{o quão próximo} estamos da vitória ou da derrota, de forma
    aproximada.  Para isto, a função deve retornar valores positivos quando o
    estado considerado representar uma chance de vitória e valores negativos
    quando o estado considerado representar uma chance de derrota.

    Dado essas informações, podemos facilmente idealizar algumas regras que a
    função heuristica \textbf{deve} considerar. No caso do jogo de damas,
    sabemos que:

    \begin{itemize}
        \item O jogador está mais perto de vencer conforme a quantidade de
            damas;
        \item Da mesma forma, está mais perto de perder conforme a quantidade
            de damas do adversário;
        \item Para ter damas, o jogador deve alcançar a última linha do
            tabuleiro, logo, quanto mais perto da última linha, maior a
            probabillidade de vencer;
        \item Da mesma forma, o jogador deve evitar que o adversário alcance a
            última linha do tabuleiro;
        \item O jogador tem mais chance de ganhar se tiver mais peças que o
            jogador adversário;
        \item Da mesma forma, ele tem mais chance de perder se tiver menos
            peças que o jogador adversário;
        \item Quando o jogador tiver a possibilidade de capturar peças do
            adversário, ele está alcançando passos da vitória. As chances
            aumentam quando o jogador consegue capturar mais de uma peça
            adversária, numa sequência e em uma mesma jogada, aumentando ainda
            mais as chances de vitória;
        \item A estratégia utilizada pelo jogador também deve considerar a
            posição em que suas peças estão em relação ao adversário, de forma
            em que este não consiga vantagem capturando várias peças do jogador
            em uma sequência;
        \item Em caso de jogadas positivas ao jogador, como por exemplo, a
            captura de peças, é desejável em que ele consiga capturar o maior
            número de peças no menor tempo possível, inclusive, é desejável que
            ele tente ao máximo de jogadas possíveis para evitar que suas peças
            sejam capturadas.
    \end{itemize}

    O algoritmo para uma função heuristica seria algo mais ou menos assim:

    \begin{minted}[breaklines=true]{python}
def heuristica(estado):
   resultado = 0
   # Nos casos a favor do computador, soma..
   resultado += quantidade_damas(estado, cor_computador)
   resultado += distancia_ultima_linha(estado, cor_computador)
   resultado += quantidade_pecas(estado, cor_computador)
   resultado += captura(estado, cor_computador) * n_pecas_capt
   
   # Nos casos a favor do adversario, decrementa..
   resultado -= quantidade_damas(estado, cor_adversario)
   resultado -= distancia_ultima_linha(estado, cor_adversario)
   resultado -= quantidade_pecas(estado, cor_adversario)
   resultado -= captura(estado, cor_adversario) * n_pecas_perdidas
   
   # Por fim, divide o resultado pelo numero de jogadas feitas de forma 
   # que seja possível priorizar um jogo que termine no menor tempo possível..
   return resultado/numero_jogadas(estado)
    \end{minted}

    Já para o caso da função utilidade, sabemos que devemos retornar um valor
    preciso, indicando apenas se o estado considerado pelo nodo folha em
    questão representa vitória ou derrota (empate não aparenta acontecer no
    jogo de damas, segundo as regras apresentadas pelo enunciado do problema).
    Para determinar isto, basta ver as condições no qual você ganha ou perde o
    jogo:

    \begin{itemize}
        \item Você ganha o jogo se o seu adversário ficar sem peças no
            tabuleiro;
        \item Você perde o jogo se você ficar sem peças para jogar.
    \end{itemize}

    Logo, a função utilidade deve se basear nesses aspectos para retornar o
    valor correspondente, portanto, positivo se representa vitória para o
    computador, negativo se representa vitória para o adversário:

    \begin{minted}{python}
def utilidade(estado):
    resultado = 0
    if numero_pecas(cor_computador) == 0:
        resultado = -1
    elif numero_pecas(cor_adversario) == 0:
        resultado = 1
    return resultado
    \end{minted}

\end{document}

