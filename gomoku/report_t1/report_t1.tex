\documentclass{article}
\usepackage[utf8]{inputenc}
\usepackage[brazil]{babel}
\usepackage[a4paper, left=20mm, right=20mm, top=20mm, bottom=20mm]{geometry}
\usepackage[colorlinks, urlcolor=blue, citecolor=red]{hyperref}
\usepackage{forest}
\usepackage{mathtools}
\usepackage{minted}

\title{\textbf{INE5430 - Inteligência Artificial \\
        \large Trabalho T1 - Parte 1}}
\author{
    Caique Rodrigues Marques \\
    {\texttt{c.r.marques@grad.ufsc.br}}
    \and
    Fernando Jorge Mota \\
    {\texttt{contato@fjorgemota.com}}
    \vspace{-5mm}
}
\date{06 de setembro de 2016}

\begin{document}

\maketitle

\section*{Função Heurística}
    Para uma máquina que queira jogar o jogo Gomoku é necessário que ela tenha
    uma boa estratégia a fim de conseguir garantir a vitória. A principal
    estratégia a se usar é considerar as possibilidades de jogadas em uma
    determinada partida, no entanto, é inviável uma máquina conseguir computar
    as várias jogadas e suas consequências possíveis, considerando o tamanho do
    tabuleiro e a quantidade de jogadas possíveis.
    
    Uma heurística é determinada para que a máquina consiga melhores
    estratégias a partir de uma estimativa, neste caso, as situações que
    complementarão para a vitória ou para a derrota.
    
    A máquina pode tanto trabalhar mais na ofensiva ou mais na defensiva,
    dependendo de como estiver a sua situação no jogo. As seguintes situações
    serão consideradas pela máquina para uma boa estratégia de jogada ofensiva:
    
    \begin{itemize}
        \item O número de peças usadas, disto, a possibilidade de formar uma
            dupla, ou uma tripla, ou uma quádrupla, ou quíntupla;
        \item A máquina deve considerar uma jogada onde se consiga formar uma
            tripla ou quádrupla ou quíntupla;
        \item Uma jogada que forma mais sequências contíguas também favorecem a
            vitória. Por exemplo, quando as peças estão dispostas onde forma um
            "L", tem duas sequências que é uma na horizontal e uma na vertical,
            aumentando as chances de vitória.
    \end{itemize}
    
    As seguintes situações são consideradas para uma jogada mais defensiva:
    
    \begin{itemize}
        \item Impedir o adversário de formar quíntuplas, quádruplas ou triplas;
        \item Considerar uma jogada em que evite que o adversário consiga
            formar sequências contíguas de peças, diminuindo as chances dele de
            conseguir, por exemplo, de formar duas triplas em forma de "L".
    \end{itemize}
    
    Ainda tem as ações que dê vantagens à máquina:
    \begin{itemize}
        \item Colocar as peças onde há mais possibilidades de formar
            sequências, por exemplo, colocar um peça no centro dá mais chances
            de sequência do que colocar nas bordas;
        \item Verificar quantas sequências já foram montadas (duplas, triplas e
            quádruplas) para uma possível quíntupla;
        \item Avaliar as sequências já formadas e ver se é possível na jogada
            em questão fechar uma sequência maior a partir de duas menores, por
            exemplo, duas duplas separadas apenas por uma casa, ao colocar uma
            peça no meio da dupla, forma uma quíntupla;
        \item A máquina também deve avaliar em quantas casas estão separadas
            duas fileiras de peças de mesma cor;
        \item O número de jogadas realizadas até o término no jogo.
    \end{itemize}
    
    Dadas todas as situações listadas, é possível atribuir ranks a cada uma, de
    forma que seja uma entrada de uma função. Com as possíveis situações à
    vista como entrada, qual será a saída gerada? A saída define o progresso da
    máquina no jogo, se está ganhando ou perdendo.
    
    Portanto, define-se uma função $v$ tal que
    \begin{gather*}
        v(p, c, b, s) = (max(p) + c \times b) \times s
    \end{gather*}
    Onde $max(p)$ corresponde ao maior número de peças contíguas, $c$
    corresponde ao número de sequências (duplas, triplas e quádruplas), $b$
    corresponde ao número de sequências formadas após a jogada (se não houver
    sequências contíguas, então $b = 1$) e $s$ corresponde ao número de peças
    contíguas formadas após a jogada (se não formar sequência, então $s = 1$)
    
    Define-se uma função $l$ tal que
    \begin{gather*}
        l(p, c, b, s) = (max(p) + c \times b) \times s
    \end{gather*}
    As variáveis têm o mesmo propósito da função $v$, mas referente ao
    adversário.
    
    A seguinte função heurística $h(x)$ recebe as possibilidades de situações
    listadas anteriormente como entrada, cada uma possuindo um valor inteiro, e
    saída é a operação entre tais valores, onde quanto maior o valor de saída,
    maior será a vantagem da máquina.
    \begin{gather*}
        h(v, l, p, j) = \frac{(v - l) \times p}{j}
    \end{gather*}
    
    Onde $v$ e $l$ correspondem às saídas das funções $v(p, c, b, s)$ e $l(p,
    c, b, s)$, respectivamente, $p$ corresponde ao número de sequências de
    peças contíguas e $j$ corresponde ao número de jogadas realizadas pela
    máquina até então.
    
\section*{Função Utilidade}
    A função de utilidade define uma certeza, que está próximo do final do jogo
    e ela define qual a melhor estratégia tomar para terminar o jogo da melhor
    maneira possível. É necessário que uma função de utilidade faça um balanço
    de tudo o que foi feito para, então, atribuir um peso final para a melhor
    decisão.
    
    Portanto, a seguinte função descreve a utilidade:
    \begin{gather*}
        u(x, seq, qt, jgds, aseq) = \frac{max(seq * qt)}{jgds + max(aseq)}
        \times x
    \end{gather*}
    Onde $seq$ corresponde ao número de peças contíguas, $qt$ corresponde ao
    número de sequências com mais peças contíguas, $jgds$ corresponde ao número
    de jogadas até então realizadas e $aseq$ corresponde ao número de peças
    contíguas do adversário tudo isso é multiplicado pelo valor de $x$ que
    define se aconteceu uma vitória ou uma derrota ou um empate. No caso de
    vitórias, o valor é positivo ($x = +1$), derrota é negativo ($x = -1$) e em
    empate é neutro ($x = 0$).
    
\section*{Estrutura de Dados}
    A estrutura de dados usada no programa é imutável. Bastante similar com um
    grafo - mais especificadamente, uma árvore - cada objeto dessa estrutura
    possui uma série de atributos que são usados durante a execução do
    programa:
    
    \begin{itemize}
        \item Uma matriz representando o tabuleiro vazio - preenchido com "+"
            internamente, para fins de representação;
        \item Uma string representando o jogador que está jogando naquele
            estado - inicialmente, pelas regras do jogo, será o jogador "O",
            que representa a peça preta;
        \item Uma string preparada para armazenar uma mensagem a ser exibida
            para o usuário. Usada principalmente em casos de erro ou avisos;
        \item Um ponteiro para o estado que gerou originalmente o estado que se
            está usando;
    \end{itemize}
    
    A partir de um determinado estado, então, essa estrutura cria cópias de si
    mesma com apenas alguns atributos modificados e um ponteiro para a versão
    que a criou. Quando é feita uma jogada, por exemplo, um novo estado é
    criado modificando apenas a matriz e o ponteiro "pai" apontando para o
    estado no qual foi feito aquela jogada. Da mesma forma, quando é imprimida
    uma mensagem, é criado um novo estado apenas configurando a nova mensagem e
    o ponteiro "pai" correspondente.
    
    No final das contas, temos, nessa estrutura, algo bem parecido com uma
    árvore/grafo: O estado inicial seria a raiz, e nenhum nodo filho seria a
    principio gerado sem que houvesse extrema necessidade. Dessa forma, é
    possível evitar uso desnecessário de memória e, pelo fato de cada objeto
    ser imutável, temos aqui uma vantagem: cada novo objeto só modifica os
    atributos modificados naquela operação. Todos os atributos que não são
    efetivamente modificados são referenciados por ponteiros no novo objeto
    criado, ajudando a economizar memória visto que a estrutura realiza cópias
    de todo atributo a cada operação feita.

\newpage
\section*{Detecção de vitória}
    A detecção de vitórias na implementação do jogo Gomoku é feita de uma forma
    simples. O tabuleiro de Gomoku é representado como uma matriz, com quinze
    linhas e quinze colunas, e, a cada peça inserida no tabuleiro, é feita uma
    checagem, partindo dessa peça, avançando 4 peças em cada uma das 8 direções
    (superior, direita, esquerda, inferior, diagonal superior esquerda,
    diagonal superior direita, diagonal inferior esquerda, diagonal inferior
    direita). Se o algoritmo detecta que um dos caminhos não foi completado
    (como no caso em que há uma peça do adversário na direção percorrido e/ou
    há um campo vazio) então o algoritmo pula para a próxima direção possível.
    
    Note que, devido a este comportamento, o número de iterações realizado é
    relativamente baixo: no melhor caso, que é quando a peça inserida não tem
    absolutamente nenhuma peça vazia, o algoritmo executa apenas 8 iterações,
    enquanto que no pior caso, que é quando o algoritmo tem 3 peças em cada
    direção possível, ele executa 32 iterações. E isso apenas quando uma peça é
    efetivamente marcada no tabuleiro, tornando todo o processamento bastante
    eficiente.
    
    Além dessa checagem a cada jogada feita, também é disponibilizado um método
    que checa cada posição no tabuleiro e retorna o primeiro que forma uma
    sequência de 5 casas. O retorno desse método, assim como no caso do método
    anterior, é o jogador que formou a sequência de 5 casas.
    
    Por fim, vale destacar que, usando esse algoritmo, é possível fazer a
    filtragem pelo jogador, de forma que o método retorna positivamente somente
    se o jogador informado tiver ganhado a partida.
    
\section*{Detecção de Maior Sequência}
    Além de disponibilizar métodos para detecção de vitória, também foram
    adicionados métodos para ajudar na detecção da maior sequência disponível
    tanto a partir de uma determinada peça quanto a partir do tabuleiro como um
    todo. 
    
    Este procedimento é feito usando o mesmo algoritmo apresentado na seção
    anterior, a única diferença básica e primordial é que ao invés de apenas
    parar quando são encontradas 5 peças em sequência, ele também vai contando
    o número de peças encontradas em cada direção e registrando o maior número
    de peças em sequência encontrado.
    
    Ao final do processamento, o maior número de peças encontrado é retornado.
    Assim como o algoritmo abordado na seção anterior, este algoritmo também
    conta com a filtragem pelo jogador, de forma que é possível obter a maior
    sequência feita por um determinado jogador.

\end{document}
